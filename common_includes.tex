% "\includegraphics" command cribs if you have more than one dot(period) in the
% filename. Include the "grffile" package in your document to avoid it. Can
% come in very handy when your image files are auto-generated.
\usepackage{grffile}

% Required to include graphics in the pdf.
\usepackage{graphics,graphicx}

% Math symbols (to get math with different styles like \mathcal, \mathbb)
\usepackage{amsmath,amssymb}

% Require this package to introduce colors in table rows and columns
\usepackage{color, colortbl}

% Table with multiple columns
%% \usepackage{multicol}

% Advanced macros specification
\usepackage{xparse}

% References
\usepackage[hyphens]{url}
\usepackage{hyperref}

% Automatically figure out the environment
% Note that hyperref has to be loaded before cleveref
% doc: http://ftp.uni-erlangen.de/ctan/macros/latex/contrib/cleveref/cleveref.pdf
\usepackage[noabbrev]{cleveref}
\newcommand{\crefrangeconjunction}{--}

% All things caption and sub-caption related
\usepackage[
  labelfont=footnotesize,
  textfont=footnotesize,
  justification=justified,
  skip=0.5\baselineskip plus 1\baselineskip minus 1\baselineskip
]{subcaption}

\usepackage[
  labelfont={bf,small},
  textfont=small,
  justification=justified,
  skip=0.5\baselineskip plus 1\baselineskip minus 1\baselineskip,
%%   font={stretch=0.85}
]{caption}

\captionsetup[subfigure]{subrefformat=simple,labelformat=simple}
\renewcommand\thesubfigure{(\alph{subfigure})}

%% In order to add spaces after macros
\usepackage{xspace}

% Helps in determining size and position of figures
\usepackage{boxedminipage2e}

% Environment to create an algorithm
% \usepackage[linesnumbered, lined, boxed]{algorithm2e}
\usepackage[ruled, linesnumbered]{algorithm2e}
\renewcommand{\listalgorithmcfname}{List of Pseudocode}
\renewcommand{\algorithmcfname}{Pseudocode}
\renewcommand{\algorithmautorefname}{pseudocode}

% Package to include eps figures in \includegraphics. The eps files are
% automatically converted to pdf to be used with pdflatex
\usepackage{epstopdf}

% Tikz -- draw figures in latex.
\usepackage{tikz}
% common tikz packages
\usetikzlibrary{backgrounds}
\usetikzlibrary{calc}
\usetikzlibrary{positioning}
\usetikzlibrary{automata}
\usetikzlibrary{shapes}
\usetikzlibrary{arrows}
\usetikzlibrary{shadows}
\usetikzlibrary{decorations.pathreplacing}
\usetikzlibrary{decorations.text}
\usetikzlibrary{shapes.multipart}
\usetikzlibrary{fit}
\usetikzlibrary{patterns}

% pgf plots are for graphs
\usepackage{pgfplots}
\usepgfplotslibrary{units}

\makeatletter
\tikzset{nomorepostaction/.code=\let\tikz@postactions\pfgutil@empty}
\makeatother
